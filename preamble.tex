\documentclass[fleqn]{article}
\usepackage{geometry}
\usepackage[format = hang, font = bf]{caption}
\usepackage{subcaption}
% The following is needed in order to make the code compatible
% with both latex/dvips and pdflatex. Added for using UML generated by MetaUML.
\ifx\pdftexversion\undefined
\usepackage[dvips]{graphicx}
\else
\usepackage[pdftex]{graphicx}
\DeclareGraphicsRule{*}{mps}{*}{}
\fi
\usepackage{array}
\newcolumntype{C}[1]{>{\centering\let\newline\\\arraybackslash}b{#1}}
\newcommand{\parcell}[2][l]{\begin{tabular}{@{}#1@{}}#2\end{tabular}}
\usepackage{pdflscape}
\usepackage{multirow}
\usepackage{graphicx}
\usepackage{floatrow}
\floatsetup[table]{capposition=top}
\usepackage{bigstrut}
\usepackage{supertabular}
\usepackage{booktabs}
\usepackage{amsmath}
\usepackage{listings}
\usepackage{multicol}
\usepackage{xcolor}
\usepackage{mathrsfs}%mathcal
\usepackage{amsfonts}%allowing \mathbb{R}
\usepackage{amssymb}
\usepackage{alltt}
\usepackage{xspace}
\usepackage{color}
\usepackage{wrapfig}%text around figure
\usepackage{lipsum}
\usepackage{tikz}
\usetikzlibrary{shapes,positioning}
\usepackage{url}
\def\UrlBreaks{\do\A\do\B\do\C\do\D\do\E\do\F\do\G\do\H\do\I\do\J\do\K\do\L\do\M\do\N\do\O\do\P\do\Q\do\R\do\S\do\T\do\U\do\V\do\W\do\X\do\Y\do\Z\do\[\do\\\do\]\do\^\do\_\do\`\do\a\do\b\do\c\do\d\do\e\do\f\do\g\do\h\do\i\do\j\do\k\do\l\do\m\do\n\do\o\do\p\do\q\do\r\do\s\do\t\do\u\do\v\do\w\do\x\do\y\do\z\do\0\do\1\do\2\do\3\do\4\do\5\do\6\do\7\do\8\do\9\do\.\do\@\do\\\do\/\do\!\do\_\do\|\do\;\do\>\do\]\do\)\do\,\do\?\do\'\do+\do\=\do\#\do\-}
\usepackage{xr}%allow cross-file references
\usepackage[breaklinks = true]{hyperref}
\lstset{
language = C, 
showspaces = false,
breaklines = true, 
tabsize = 2, 
extendedchars = false, 
basicstyle = {\ttfamily \footnotesize}, 
showstringspaces=false,
keywordstyle=\color{blue!70}, 
commentstyle=\color{gray},
rulesepcolor=\color{red!20!green!20!blue!20}, 
numberstyle=\color[RGB]{0,192,192},
stringstyle=\color{red},
escapeinside={ )},
moredelim=[is][\underbar]{(**}{**)},
mathescape
}
\mathchardef\myhyphen="2D